% ---- add headings for #### and ##### ----

\usepackage{titlesec}

% Define formatting for \paragraph (level 4 heading)
\titleformat{\paragraph}
  {\normalfont\normalsize\bfseries}{\theparagraph}{1em}{}
\titlespacing*{\paragraph}
  {0pt}{3.25ex plus 1ex minus .2ex}{1.5ex plus .2ex}

% Define formatting for \subparagraph (level 5 heading)
\titleformat{\subparagraph}
  {\normalfont\normalsize\bfseries\itshape}{\thesubparagraph}{1em}{}
\titlespacing*{\subparagraph}
  {0pt}{3.25ex plus 1ex minus .2ex}{1.5ex plus .2ex}

% ---- more spacing for young learners ----

% Increase overall line spacing for a more open, readable layout
\usepackage{setspace}
% \setstretch{1.2}

% % Increase font sizes for section headings to be more child-friendly
% \usepackage{sectsty}
% \sectionfont{\fontsize{16}{20}\selectfont\bfseries}
% \subsectionfont{\fontsize{14}{18}\selectfont\bfseries}
% \subsubsectionfont{\fontsize{12}{16}\selectfont\bfseries}
% % Level 4 and 5 headers are already set up in your file

% % Adjust spacing before and after section titles (if desired)
% \usepackage{titlesec}
% \titlespacing*{\section}{0pt}{2.5ex plus 1ex minus .2ex}{2ex plus .2ex}
% \titlespacing*{\subsection}{0pt}{2.0ex plus 1ex minus .2ex}{1.5ex plus .2ex}
% \titlespacing*{\subsubsection}{0pt}{1.5ex plus 1ex minus .2ex}{1.2ex plus .2ex}

% % Use fancyhdr to create a simple, larger header and footer (optional)
% \usepackage{fancyhdr}
% \pagestyle{fancy}
% \fancyhf{}
% \fancyhead[C]{\fontsize{12}{14}\selectfont Second Grade Mathematics}
% \fancyfoot[C]{\fontsize{12}{14}\selectfont \thepage}
% \renewcommand{\headrulewidth}{0pt}

% Optionally adjust list spacing for clarity
% \usepackage{enumitem}
% \setlist{itemsep=1ex, topsep=1ex}


% ---- Custom styling for block quotes ----
\usepackage{mdframed}
\usepackage{xcolor}

\renewenvironment{quote}
  {\begin{mdframed}[backgroundcolor=gray!10, linecolor=gray!50, leftline=true, topline=false, bottomline=false, rightline=false, innerleftmargin=10pt, innerrightmargin=10pt, innertopmargin=10pt, innerbottommargin=10pt, skipabove=10pt, skipbelow=10pt]}
  {\end{mdframed}}

% % Add extra vertical space before and after horizontal rules
% \let\oldhrule\hrule
% \renewcommand{\hrule}{\par\vspace{1em}\oldhrule\par\vspace{1em}}
% \makeatletter
% \let\oldhrule\hrule
% \renewcommand{\hrule}{\par\addvspace{1em}\oldhrule\addvspace{1em}\par}
% \makeatother

% \newcommand{\myhrule}{\par\vspace*{1em}\hrule\vspace*{1em}\par}

% \usepackage{etoolbox}
% \makeatletter
% \let\oldhrule\hrule
% \AtBeginDocument{%
%   \renewcommand{\hrule}{\par\addvspace{1em}\oldhrule\addvspace{1em}\par}%
% }
% \makeatother

\newcommand{\myhrule}{%
  \par\vspace*{1em}%
  {\centering\rule{0.6\linewidth}{0.4pt}\par}%
  \vspace*{1em}%
}

% Plots
\usepackage{pgfplots}
\pgfplotsset{width=10cm,compat=1.9}
\usepgfplotslibrary{statistics}

% \newenvironment{centeredtikz}
%   {\par\vspace{2em}\begin{center}\begin{tikzpicture}}
%   {\end{tikzpicture}\end{center}\par\vspace{2em}}

% Fing emojis!!
% \usepackage{fontspec}
% \setmainfont{DejaVu Serif} % or your preferred main font
% \newfontfamily\emojifont{NotoColorEmoji}[
%   Renderer=Harfbuzz,
%   Path = /usr/share/fonts/truetype/noto/,
%   Extension = .ttf,
%   Scale = MatchLowercase
% ]
% \newcommand{\emoji}[1]{{\emojifont #1}}

% \documentclass[twoside]{article}
\usepackage{fancyhdr}
\pagestyle{fancy}
\fancyhf{} % clear all header and footer fields
\fancyfoot[LE,RO]{\thepage} % left footer on even pages, right footer on odd pages
\renewcommand{\headrulewidth}{0pt}
\renewcommand{\footrulewidth}{0pt}
\usepackage{tikz}
\usepackage{pgfmath}
\usepackage{ifthen}

\newcommand{\hexagram}[6]{%
\begin{center}
  \begin{tikzpicture}[scale=0.8]
  \def\hexone{#1}%
  \def\hextwo{#2}%
  \def\hexthree{#3}%
  \def\hexfour{#4}%
  \def\hexfive{#5}%
  \def\hexsix{#6}%

    % Loop over line index (from top: 6 to bottom: 1) with associated value
    \foreach \i/\val in {6/\hexsix,5/\hexfive,4/\hexfour,3/\hexthree,2/\hextwo,1/\hexone} {
      \pgfmathparse{-(7-\i)*0.7}%
      \let\y\pgfmathresult
      \ifnum\val=1
        % Yang: Solid line
        \draw (0,\y) -- (4,\y);
      \else
        % Yin: Broken line (two segments)
        \draw (0,\y) -- (1.5,\y);
        \draw (2.5,\y) -- (4,\y);
      \fi
    }
  \end{tikzpicture}%
\end{center}
}
